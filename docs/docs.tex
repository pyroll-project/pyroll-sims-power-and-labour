%! Author = Christoph Renzing
%! Date = 10.05.2022

% Preamble
\documentclass[11pt]{PyRollDocs}
\usepackage{mathtools}
\addbibresource{refs.bib}

% Document
\begin{document}

    \title{The Sims Power and Labour PyRoll Plugin}
    \author{Christoph Renzing}
    \date{\today}

    \maketitle

    This plugin provides the roll force and roll torque model developed by R.~B.~Sims \textcite{Sims1954}.
    The basic equations are derived from classic strip theory with suitable simplifications suitable for hot rolling (e.g. static friction, neglect of elastic material behaviour).
    Usage of the equations for groove rolling is only valid, when using a equivalent rectangle approach.
    Heights used for calculation and therefore equivalent values for a equivalent flat pass.
    The work by Sims, implements a flattened roll radius $R_1$ which is to be calculated through suitable models.
    Furthermore, the used variable $h$ is the height of the equivalent flat workpiece, $k_{f,m}$ represents the mean flow stress of the material and $\epsilon$ the reduction of the equivalent pass in height direction.
    The indices 0 and 1 denote the incoming and exiting profile.


    \section{Model approach}\label{sec:model-approach}

    To calculate the roll force in hot rolling, the following equation was developed:

    \begin{equation}
        F_{Roll} = k_{f, m} \sqrt{R \Delta h} Q_{Force}(\frac{R}{h_1}, \epsilon_h)
        \label{eq:sims-force-simple}
    \end{equation}

    The function $Q_{Force}$ is a which combines the influence of the reduction of the pass.
    This function depends on the height of the workpiece at the neutral line $h_{n}$.
    Calculation of the neutral line angle is done through equation~\ref{eq:neutral-line-angle} and the height of the workpiece dependent on the roll angle $\alpha$ is express with equation ~\ref{eq:roll-gap-height}.

    \begin{subequations}
        \begin{equation}
            \begin{multlined}
                Q_{Force} = \frac{\pi}{2} \sqrt{\frac{1 - \epsilon}{\epsilon}} \arctan\left( \sqrt{\frac{\epsilon}{1 - \epsilon}} \right) - \frac{\pi}{4} - \sqrt{\frac{1 - \epsilon}{\epsilon}} \sqrt{\frac{R_1}{h_1}} \log\left( \frac{h_ {n}}{h_1} \right) + \\
                \frac{1}{2} \sqrt{\frac{1 - \epsilon}{\epsilon}} \sqrt{\frac{R_1}{h_1}} \log \left( \frac{1}{1 - \epsilon} \right)
            \end{multlined}
            \label{eq:sims-force-function}
        \end{equation}
        \begin{equation}
            \alpha_{n} = \frac{\tan \left( \frac{1}{8} (4 \arctan (\frac{\epsilon}{1 - \epsilon}) + \frac{\pi \log (1 - \epsilon)}{\sqrt{\frac{R_1}{h_1}}} )\right)}{\sqrt{\frac{R_1}{h_1}}}
            \label{eq:neutral-line-angle}\\
        \end{equation}
        \begin{equation}
            h(\alpha) = h_1 + 2 R ( 1 - \cos(\alpha))
            \label{eq:roll-gap-height}
        \end{equation}
        \begin{equation}
            k_{f,m} = \frac{k_{f, 0} + 2 k_{f, 1}}{3}
            \label{eq:mean-flow-stress}
        \end{equation}
    \end{subequations}

    As for the roll torque $M_{roll}$, Sims developed a similar equation.

    \begin{equation}
        M_{roll} = 2 R^2 k_{f, m} Q_{Torque}
        \label{eq:sims-torque}
    \end{equation}

    The function $Q_{Torque}$ depends on the entry angle of the equivalent flat pass $\alpha_0$ which is calculated from equation~\ref{eq:entry-angle}.
    Furthermore, $L_d$ is the contact length of the equivalent flat pass.

    \begin{subequations}
        \begin{equation}
            Q_{Torque} = \frac{a_0}{2} - \alpha_n
            \label{eq:sims-torque-function}\\
        \end{equation}
        \begin{equation}
            \alpha_0 =  \arcsin \left( \frac{L_d}{R} \right)
            \label{eq:entry-angle}
        \end{equation}
    \end{subequations}


    \section{Usage instructions}\label{sec:usage-instructions}

    The plugin can be loaded under the name \texttt{pyroll\_sims\_power\_and\_labour}.

    An implementation of the \lstinline{roll_force} and \lstinline{roll_tourque} hook on \lstinline{RollPass} and \lstinline{Roll} is provided,
    calculating the roll force and torque using the equivalent rectangle approach.

    Several additional hooks on \lstinline{RollPass} are defined, which are used in spread calculation, as listed in \autoref{tab:hookspecs}.
    Base implementations of them are provided, so it should work out of the box.
    For \lstinline{sims_force_function} and \lstinline{sims_torque_function} the equations~\ref{eq:sims-force-function} and~\ref{eq:sims-torque} are implemented.
    Provide your own hook implementations or set attributes on the \lstinline{RollPass} instances to alter the spreading behavior.

    \begin{table}
        \centering
        \caption{Hooks specified by this plugin.}
        \label{tab:hookspecs}
        \begin{tabular}{ll}
            \toprule
            Hook name                                 & Meaning                                     \\
            \midrule
            \texttt{equivalent\_height\_change}       & Equivalent height change $\Delta h$         \\
            \texttt{equivalent\_reduction}            & Equivalent reduction $\epsilon$             \\
            \texttt{equivalent\_neutral\_line\_angle} & Equivalent angle of neutral line $\alpha_0$ \\
            \bottomrule
        \end{tabular}
    \end{table}

    \printbibliography


\end{document}